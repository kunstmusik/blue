\section{NoteProcessors}\label{noteProcessors}

NoteProcessors are used in conjunction with soundObjects, and are used
post-generation of the soundObject's noteList. They are used to modify
values within the noteList.

NoteProcessors can be added via the soundObject property dialog. When a
soundObject is selected on the timeline, and if the soundObject supports
noteProcessors, you can add, remove, push up, or push down
noteProcessors on the property dialog.

NoteProcessors are applied after the notes of the soundObject are
generated and before time behavior is applied. Processing starts with
the first NoteProcessor in the chain and the results of that are passed
down the chain.


\subsection{Add Processor}\label{addProcessor}

Parameters: pfield, value

The AddProcessor takes two parameters, one for pfield (positive integer
\textgreater{} 0) and one for value (any number). When applied, it will
add the user-defined value to the set pfield for all notes.

For example, if you have a SoundObject with notes for an instrument that
uses p4 as its amplitude. If you have values for p4 within the range of
78 and 82, such as the following score:

\begin{verbatim}
i1 0 2 78
i1 + . 80
i1 + . 81
i1 + . 82
  
\end{verbatim}

If an AddProcessor is used that was set to 4.4 for value 4 for pfield,
your notes afterwards would be moved to the range of 82.4 to 86.4:

\begin{verbatim}
i1  0.0 0.5 82.4
i1  0.5 0.5 84.4
i1  1.0 0.5 85.4
i1  1.5 0.5 86.4
  
\end{verbatim}

(The p2 and p3 times above are post-processing for a 2 second duration
SoundObject with time behavior set to scale.)


\subsection{Equals Processor}\label{equalsProcessor}

Parameters: pfield, value

Sets user-given pField of all notes in soundObject to user-given value,
i.e. set all p4's to value "440", or set all p6's to value
"/work/audio/wav/mySample2.wav". The following score:

\begin{verbatim}
i1 0 2 8.00
i1 + . 8.04
i1 + . 8.07
i1 + . 9.00    
\end{verbatim}

If used with an EqualsProcessor with value 7.00 and pfield 4, would
result in the following score:

\begin{verbatim}
i1  0.0 0.5 7.00
i1  0.5 0.5 7.00
i1  1.0 0.5 7.00
i1  1.5 0.5 7.00
\end{verbatim}

(The p2 and p3 times above are post-processing for a 2 second duration
SoundObject with time behavior set to scale.)

Tip: One can use this NoteProcessor to quickly try testing a score with
another insturment. To do this, use this NoteProcessor to reassign p1.


\subsection{Inversion Processor}\label{inversionProcessor}

Parameters: pfield, value

This NoteProcessor flips all values in designated pfield about an axis
(value). The following score:

\begin{verbatim}
i1 0 2 80
i1 + . 85
i1 + . 95
i1 + . 100
\end{verbatim}

If processed with an InversionProcessor with value 90 and pfield 4 would
result in:

\begin{verbatim}
i1  0.0 0.5 100.0
i1  0.5 0.5 95.0
i1  1.0 0.5 85.0
i1  1.5 0.5 80.0    
\end{verbatim}

(The p2 and p3 times above are post-processing for a 2 second duration
SoundObject with time behavior set to scale.)


\subsection{LineAdd Processor}\label{lineAddProcessor}

Parameters: pfield, LineAdd String

The LineAdd Processor adds values to a user-defined pfield. This
noteProcessor differs from the AddProcessor in that the value added is
variable over time. The LineAdd String is a set of beat/value pairs that
are like the breakpoints on curve. The following score:

\begin{verbatim}
i1 0 2 80
i1 + . 80
i1 + . 80
i1 + . 80
\end{verbatim}

If processed with a LineAdd Processor with LineAddString set to "0 0 6
3" and pfield set to 4, would result in:

\begin{verbatim}
i1  0.0 0.5 80.0
i1  0.5 0.5 81.0
i1  1.0 0.5 82.0
i1  1.5 0.5 83.0
\end{verbatim}

(The p2 and p3 times above are post-processing for a 2 second duration
soundObject with time behavior set to scale.)

The LineAddString can be interpreted to mean: start at beat 0 with value
0, and by beat 6, arrive at value 3. For each note, the LineAddString
will find the note's beat and compare against the LineAddString to see
how much should be added. The user should be careful to remember that
SoundObjects do not apply scaling of score until after processing with
noteProcessors. Therefore, the beat values given in the LineAddString
were to be applied against the beat values of the original score.


\subsection{LineMultiply Processor}\label{lineMultiplyProcessor}

Parameters: pfield, LineMultiply String

The LineMultiply Processor multiplies values in a user-defined pfield.
This noteProcessor differs from the Multiply Processor in that the value
muliplied by is variable over time. The LineMultiply String is a set of
beat/value pairs that are like the breakpoints on curve. The following
score:

\begin{verbatim}
i1 0 2 80
i1 + . 80
i1 + . 80
i1 + . 80
\end{verbatim}

If processed with a LineMultiple Processor with LineMultiplyString set
to "0 0 6 2" and pfield set to 4, would result in:

\begin{verbatim}
i1  0.0 0.5 0.0
i1  0.5 0.5 53.333336
i1  1.0 0.5 106.66667
i1  1.5 0.5 160.0
\end{verbatim}

(The p2 and p3 times above are post-processing for a 2 second duration
soundObject with time behavior set to scale.)

The LineMultiplyString can be interpreted to mean: start at beat 0 with
value 0, and by beat 6, arrive at value 2. For each note, the
LineMultiplyString will find the note's beat and compare against the
LineMultiplyString to see how much should be multiplied. The user should
be careful to remember that SoundObjects do not apply scaling of score
until after processing with noteProcessors. Therefore, the beat values
given in the LineMultiplyString were to be applied against the beat
values of the original score.


\subsection{Multiply Processor}\label{multiplyProcessor}

Parameters: value, pfield

The MultiplyProcessor works like the addProcessor, but multiplies the
given pfield by the value. The following score:

\begin{verbatim}
i1 0 2 80
i1 + . 80
i1 + . 80
i1 + . 80
\end{verbatim}

If processed with a MultiplyProcessor with value 2 and pfield 4 would
result in:

\begin{verbatim}
i1  0.0 0.5 160.0
i1  0.5 0.5 160.0
i1  1.0 0.5 160.0
i1  1.5 0.5 160.0
\end{verbatim}

(The p2 and p3 times above are post-processing for a 2 second duration
soundObject with time behavior set to scale.)


\subsection{Pch Add Processor}\label{pchAddProcessor}

Parameters: pfield, value

The PchAddProcessor works like the AddProcessor, but is used with
pfields written in Csound pch notation(i.e. 8.00, 7.11, 9.03). The value
field, unlike the AddProcessor, is a whole-number, representing the
number of scale degrees in which to add to the pch value. If a
soundObject has notes where p5 is a pch value, and you the notes have
values of 8.00, 8.04, 8.07, and 9.00:

\begin{verbatim}
i1 0 2 8.00
i1 + . 8.04
i1 + . 8.07
i1 + . 9.00
\end{verbatim}

If a PchAddProcessor with value 7 and pfield 5 is used, after
processing, the score would result in:

\begin{verbatim}
i1  0.0 0.5 8.07
i1  0.5 0.5 8.11
i1  1.0 0.5 9.02
i1  1.5 0.5 9.07
\end{verbatim}

(The p2 and p3 times above are post-processing for a 2 second duration
soundObject with time behavior set to scale.)

If you had used a regular AddProcessor and tried to transpose 7 scale
degrees down using -.07, you would have gotten 7.93, 7.97, 8.00, and
8.97, and the first few notes would have really be equal to 14.11,
15.01, and 8.00 in pch notation. This is due to the way that Csound pch
notation works to hold the scale degree, from 0-11, in the two fields to
the right of the period.


\subsection{Pch Inversion Processor}\label{pchInversionProcessor}

Parameters: value, pfield

This noteProcessor flips all values in designated pfield about an axis
(value). The values that are read in the given pfield will be
interpreted as Csound pch format. The following score:

\begin{verbatim}
i1 0 2 8.00
i1 + . 8.04
i1 + . 8.07
i1 + . 9.00
\end{verbatim}

If processed with a PchInversionProcessor with value 8.06 and pfield 4
would result in:

\begin{verbatim}
i1  0.0 0.5 9.0
i1  0.5 0.5 8.08
i1  1.0 0.5 8.05
i1  1.5 0.5 8.0    
\end{verbatim}

(The p2 and p3 times above are post-processing for a 2 second duration
soundObject with time behavior set to scale.)


\subsection{Python Processor}\label{pythonProcessor}

Parameters: code

Allows user to write python code to process NoteList. The code is run
using Jython, the Java implementation of Python that is included with
Blue. Users using the PythonProcessor should be aware of Jython's
implementation details: most classes and modules from CPython are
included, users can not import modules with native libraries, and users
can use any Java class that exists in the classpath.

When Blue generates notes from a SoundObject, the SoundObject first
generates its Notes as a NoteList, then NoteProcessors are called one at
a time to process that NoteList in place. For the PythonProcessor,
before any of the user-written code is run, the generated NoteList is
first added and defined in the memory space as "noteList". From there,
the script for the PythonProcessor should treat noteList as the list of
Note objects that have been generated. The NoteList and Note objects are
the very same and have the same methods and properties as those in the
Java code, so one must work with them the same way as one would in Java.

The following code shows an example of the Python Processor. The code
first imports the random module, then for every note in the noteList it
creates a random space variable between the values of -1 and 1, then
assigns that to pfield 7 of the note. (This example code is used in the
blue/examples/noteProcessors/pythonProcessor.blue and can be run within
Blue to hear the example).

\begin{verbatim}
import random

for i in noteList:
  newVal = str((2 * random.random()) - 1)
  i.setPField(newVal, 7)
\end{verbatim}


\subsection{Random Add Processor}\label{randomAddProcessor}

Parameters: pfield, min, max, seedUsed, seed

The RandomAddProcessor generates a random value between min and max and
adds it to the designated pfield for all notes. A random value is
generated for each note.

The RandomAddProcessor expects either a positive or negative float value
for the min and max, and a postive integer for the pfield. The following
score:

\begin{verbatim}
i1 0 2 80
i1 + . 80
i1 + . 80
i1 + . 80
\end{verbatim}

when processed with an RandomAddProcessor with min set to 0.0, max set
to 1.0, and pfield set to 4, resulted in the following on one pass:

\begin{verbatim}
i1  0.0 0.5 80.92294
i1  0.5 0.5 80.50539
i1  1.0 0.5 80.112495
i1  1.5 0.5 80.93934
\end{verbatim}

(The p2 and p3 times above are post-processing for a 2 second duration
soundObject with time behavior set to scale.)

The above is an example of random values, and because it is random, the
results will be different on another processing pass. .

If seedUsed is set to "true", the random number generator will be
initialized using the given seed value. The seed value must be a valid
long integer value (-2\^{}63, 2\^{}63-1).

Hint: One useful way to use this noteProcessor is as a way to "humanize"
velocity or pitch values values.


\subsection{Random Multiply Processor}\label{randomMultiplyProcessor}

Parameters: pfield, min, max

The RandomMultiplyProcessor takes a pfield to apply the random multiply
to, as well as a min and max for the boundaries of the random values.
When applied, it will multiply a random value for each note in the
assigned pfield, using a new random value per note.

The RandomMultiplyProcessor expects either a positive or negative float
value for the min and max, and a postive integer for the pfield. The
following score:

\begin{verbatim}
i1 0 2 80
i1 + . 80
i1 + . 80
i1 + . 80
\end{verbatim}

when processed with an RandomMultiplyProcessor with min set to 1.0, max
set to 2.0, and pfield set to 4, results in the following:

\begin{verbatim}
i1  0.0 0.5 85.553246
i1  0.5 0.5 148.94167
i1  1.0 0.5 125.57565
i1  1.5 0.5 97.00755
\end{verbatim}

(The p2 and p3 times above are post-processing for a 2 second duration
soundObject with time behavior set to scale.)

If seedUsed is set to "true", the random number generator will be
initialized using the given seed value. The seed value must be a valid
long integer value (-2\^{}63, 2\^{}63-1).

The above is an example of random values, and because it is random, the
results will be different on another processing pass.


\subsection{Retrograde Processor}\label{retrogradeProcessor}

Parameters: none

Reverses all the generated notes in time. The following score:

\begin{verbatim}
i1 0 2 1
i1 + . 2
i1 + . 3
i1 + . 4
\end{verbatim}

If processed with a RetrogradeProcessor would result in:

\begin{verbatim}
i1  1.5 0.5 1
i1  1.0 0.5 2
i1  0.5 0.5 3
i1  0.0 0.5 4
\end{verbatim}

which if re-sorted by start time would result in:

\begin{verbatim}
i1  0.0 0.5 4
i1  0.5 0.5 3
i1  1.0 0.5 2
i1  1.5 0.5 1
\end{verbatim}

(The p2 and p3 times above are post-processing for a 2 second duration
soundObject with time behavior set to scale.)


\subsection{Rotate Processor}\label{rotateProcessor}

parameters: noteIndex

The RotateProcessor rotates the order of the notes, making the noteIndex
the first note. For example a noteIndex of 2 means "make the second note
of the notes generated the first, shifting the first note to the end".
With a noteIndex of 3, the following score:

\begin{verbatim}
i1 0 2 1
i1 2 2 2
i1 4 2 3
i1 5 2 4
i1 6 2 5
\end{verbatim}

will become:

\begin{verbatim}
i1 0 2 3
i1 2 2 4
i1 4 2 5
i1 5 2 1
i1 6 2 2
\end{verbatim}

As the third note now becomes the first. If a negative number is given,
the number will count backwards as to what note should now become the
first note. With a noteIndex of -2 (meaning, "Second note from the
end"), the following score:

\begin{verbatim}
i1 0 2 1
i1 2 2 2
i1 4 2 3
i1 5 2 4
i1 6 2 5
\end{verbatim}

Will become:

\begin{verbatim}
i1 0 2 4
i1 2 2 5
i1 4 2 1
i1 5 2 2
i1 6 2 3
\end{verbatim}

Start times of notes are all modified such that the notes that are
pushed to the end of list will start at the end time of the last note
before it. For the above examples, the duration of the note list ast a
whole did not change at all. However, when there are overlapping notes
as in the following example, where note index of -3 was used:

\begin{verbatim}
i1  0   5.0 0   4   
i1  2   5.0 1   4   
i1  4   5.0 2   4   
i1  6   5.0 3   4   
i1  8   5.0 4   4   
i1  10  5.0 5   4   
i1  12  5.0 6   4   
i1  14  5.0 7   4   
i1  16  5.0 8   4   
i1  18  5.0 9   4   
\end{verbatim}

The resultant scores duration has changed as shown below:

\begin{verbatim}
i1  0.0 5.0 7   4   
i1  2.0 5.0 8   4   
i1  4.0 5.0 9   4   
i1  9.0 5.0 0   4   
i1  11.0    5.0 1   4   
i1  13.0    5.0 2   4   
i1  15.0    5.0 3   4   
i1  17.0    5.0 4   4   
i1  19.0    5.0 5   4   
i1  21.0    5.0 6   4   
\end{verbatim}

Please be aware of this behavior when using this NoteProcessor with
scores that have overlapping notes.


\subsection{SubList Processor}\label{sublistProcessor}

Parameters: start, end

The SubListProcessor will cut out notes from the soundObject's generated
noteList. An example of it's use may be that you have a 12-tone row as
soundObject in the soundObject library, and you're using instances of it
as the basis of your work. You may only want to use notes 1-3 of the
row, so you would use the SublistProcessor with a start of 1 and an end
of 3.

The SubListProcessor will cut out notes, then translate them to start at
the start of the soundObject, and then scale them so that they take up
the duration of the soundObject. If you had a five note soundObject with
all notes have a duration of 1 second, all starting one after the other,
with the soundObject starting at 0 seconds on the timeline, and if you
used a SubListProcessor with start of 1 and end of 4, you'd end up with
four notes being generated(the first four from the original
soundObject), starting a 0 seconds on the timeline, with each notes
duration lasting 1.25 seconds, each starting one right after the other.

The following score:

\begin{verbatim}
i1 0 2 1
i1 + . 2
i1 + . 3
i1 + . 4
\end{verbatim}

when processed with a SubList processor with start 2 and end 3, would
result in:

\begin{verbatim}
i1  0.0 1   2
i1  1.0 1   3
\end{verbatim}

(The p2 and p3 times above are post-processing for a 2 second duration
soundObject with time behavior set to scale.)


\subsection{Switch Processor}\label{switchProcessor}

parameters: pfield1, pfield2

A Switch Processor switches pfield1 with pfield2 for all notes in
SoundObject. It is useful in conjunction with RetrogradeProcessor when
reversing notes that have start and end values in pfields, i.e.

\begin{verbatim}
;inum start dur   start   end
i1    0     1     8.04    8.00
i1    1     2     8.00    8.07
\end{verbatim}

with just retrograde processor becomes:

\begin{verbatim}
;inum start dur   start   end
i1    0     2     8.00    8.07
i1    2     1     8.04    8.00
\end{verbatim}

with retrograde and switch on p4 and p5 becomes:

\begin{verbatim}
;inum start dur   start   end
i1    0     2     8.07    8.00
i1    2     1     8.00    8.04
\end{verbatim}


\subsection{Time Warp Processor}\label{timewarpProcessor}

parameters: TimeWarpString

Warps time in the same way as Csound t-statement, but does not require
"t" to be used. Statements are in alternating pairs of beat number and
tempo.

From the Csound Manual:

Time and Tempo-for-that-time are given as ordered couples that define
points on a "tempo vs. time" graph. (The time-axis here is in beats so
is not necessarily linear.) The beat-rate of a Section can be thought of
as a movement from point to point on that graph: motion between two
points of equal height signifies constant tempo, while motion between
two points of unequal height will cause an accelarando or ritardando
accordingly. The graph can contain discontinuities: two points given
equal times but different tempi will cause an immediate tempo change.

Motion between different tempos over non-zero time is inverse linear.
That is, an accelerando between two tempos M1 and M2 proceeds by linear
interpolation of the single-beat durations from 60/M1 to 60/M2.

The first tempo given must be for beat 0.

\begin{itemize}
\item
  Beat values for beat/tempo pairs should related to the score *before*
  any time behavior is applied. For example, for the following score:

\begin{verbatim}
i1 0 1 2 3 4
i1 1 1 3 4 5 
i1 2 1 3 4 5 
i1 3 1 3 4 5 
\end{verbatim}

  if it is in a GenericScore SoundObject of duration 20, if you want the
  tempo to decrease in half by the last note, you would enter a value
  for the processor as "0 60 3 30" and not "0 60 20 30"
\item
  If you're using a time behavior of "Repeat", remember that time
  behavior is applied *after* noteProcessors, and the resulting score
  will be a time warped score repeated x times and *NOT* a score
  repeated x time and then timewarped
\item
  Time Warping, when used with a time behavior of "Scale", be aware that
  estimating the final tempo of the object may be tricky, as the scaling
  will alter the duration of notes.
\end{itemize}


\subsection{Tuning Processor}\label{tuningProcessor}

Parameters: pfield, baseFrequency, scalaFile

Converts Blue PCH notation to frequency according to scale values in a
Scala scale file. The scale will default to 12TET wheen TuningProcessor
is initially created. The file selector for choosing a Scala .scl file
will default to user's .blue directory, under the scl subdirectory. It
is advised that users download the 3000+ scale archive from the Scala
website at: http://www.huygens-fokker.org/scala/ and place them in the
.blue/scl directory or whever they find convenient.

Base frequency determines what 8.00 should be set to, defaulting to
middle-C below A440 (same as in Scala). Input for the noteProcessor
should be:

\begin{verbatim}
oct.scaleDegree
\end{verbatim}

where oct is equal to octave, and scale degree equal to what degree of
the scale to use. The output will be frequency values, so instruments
should be set to accept values as frequency.

\begin{itemize}
\item
  Fractional scaleDegree's are not supported.
\item
  For the scaleDegree, the Blue PCH does not work exactly like Csound
  pch notation. In Csound pch, "8.01" would be different than "8.1",
  while for the Tuning processor, there is no difference. The tuning
  processor takes everything from the right side of the decimal and
  converts that to an integer to figure out the scale degree.
\item
  If you enter in a scale degree higher than the number degrees in the
  scale, it will be converted as if the octave is raised. For example,
  in a 19 tone scale, an input of "8.21" would get converted to "9.1".
\end{itemize}
