\section{Shortcuts}\label{shortcuts}

\begin{longtable}[]{@{}ll@{}}
\caption{General Shortcuts}\tabularnewline
\toprule
Shortcuts & Description\tabularnewline
\midrule
\endfirsthead
\toprule
Shortcuts & Description\tabularnewline
\midrule
\endhead
ctrl-1 & brings the score tab into focus\tabularnewline
ctrl-2 & brings the orchestra tab into focus\tabularnewline
ctrl-3 & brings the tables tab into focus\tabularnewline
ctrl-4 & brings the globals tab into focus\tabularnewline
ctrl-5 & brings the project properties tab into focus\tabularnewline
ctrl-6 & brings the soundfile tab into focus\tabularnewline
alt-right & brings the next manager tab into focus\tabularnewline
alt-left & brings the previous manager tab into focus\tabularnewline
F9 & start/stop a render (equivalent to pressing the render/stop
button)\tabularnewline
ctrl-g & generate a CSD file\tabularnewline
ctrl-shift-g & generate a CSD to screen (for previewing)\tabularnewline
ctrl-o & open a work file\tabularnewline
ctrl-s & save work file (must use "Save as" from file menu if a new work
file)\tabularnewline
ctrl-w & close the current work file\tabularnewline
alt-F4 & close blue\tabularnewline
\bottomrule
\end{longtable}

\begin{longtable}[]{@{}ll@{}}
\caption{Rendering}\tabularnewline
\toprule
Shortcuts & Description\tabularnewline
\midrule
\endfirsthead
\toprule
Shortcuts & Description\tabularnewline
\midrule
\endhead
F9 & Render project using project's real-time render
options\tabularnewline
shift-F9 & Render to Disk and Play using project's disk render options
and playing with blue's builtin sound file player\tabularnewline
ctrl-shift-F9 & Render to Disk using project's disk render
options\tabularnewline
\bottomrule
\end{longtable}

\begin{longtable}[]{@{}ll@{}}
\caption{Score Timeline}\tabularnewline
\toprule
Shortcuts & Description\tabularnewline
\midrule
\endfirsthead
\toprule
Shortcuts & Description\tabularnewline
\midrule
\endhead
ctrl-c & copy selected soundObject(s)\tabularnewline
ctrl-x & cut selected soundObject(s)\tabularnewline
ctrl-d & duplicate selected soundObject(s) and place directly after
originals\tabularnewline
ctrl-r & the repeat selected SoundObjects by copying and placing one
after the other n number of times where n is a number value entered by
the user (user is prompted with a dialog to enter number of times to
repeat)\tabularnewline
ctrl-click & paste soundObject(s) from buffer where
clicked\tabularnewline
shift-click & on timeline, paste soundObjects from buffer as a
PolyObject where clicked, if only one sound object is in the buffer it
will be pasted as the type it is and not a PolyObject\tabularnewline
shift-click & when selecting soundObjects, adds soundObject to selected
if not currently selected and vice-versa\tabularnewline
double-click & if selecting on timeline, select all soundObjects on
layer where mouse clicked\tabularnewline
ctrl-t & show quick time dialog\tabularnewline
right click & pops up a popup menu for adding soundObjects and other
tasks\tabularnewline
left click + drag & creates a selection marquee for selecting multiple
soundObjects (when Score mode)\tabularnewline
alt-1 & switch to Score mode\tabularnewline
alt-2 & switch to Single Line mode\tabularnewline
alt-3 & switch to Mult Line mode\tabularnewline
left & nudge selected soundObjects one pixel to the left\tabularnewline
right & nudge selected soundObject one pixel to the right\tabularnewline
shift-left & nudge selected soundObjects ten pixels to the
left\tabularnewline
shift-right & nudge selected soundObjects ten pixels to the
right\tabularnewline
up & move selected soundObjects up one layer\tabularnewline
down & move selected soundObjects down one layer\tabularnewline
ctrl-left & decrease horizontal zoom\tabularnewline
ctrl-right & incrase horizontal zoom\tabularnewline
\bottomrule
\end{longtable}

\begin{longtable}[]{@{}ll@{}}
\caption{Orchestra Manager}\tabularnewline
\toprule
Shortcuts & Description\tabularnewline
\midrule
\endfirsthead
\toprule
Shortcuts & Description\tabularnewline
\midrule
\endhead
ctrl-left click & if on the column header of the instruments table, will
enable/disable all instruments\tabularnewline
\bottomrule
\end{longtable}

\begin{longtable}[]{@{}ll@{}}
\caption{In a text box}\tabularnewline
\toprule
Shortcuts & Description\tabularnewline
\midrule
\endfirsthead
\toprule
Shortcuts & Description\tabularnewline
\midrule
\endhead
right click & pops up a popup menu that contains entries for opcodes as
well as user-defined entries from the code repository\tabularnewline
ctrl-space & brings up a dialog that shows all possible opcode matches
for the current word being typed in (code completion)\tabularnewline
shift-F1 & if cursor is within a word that is an opcode, attempts to
look up manual entry for that opcode\tabularnewline
shift-F2 & if cursor is within a word that is an opcode, attempts to
find a manual example CSD for that opcode and if found opens a dialog
showing the example\tabularnewline
ctrl-; & comment out line or selected lines (prepends ";" to every
line)\tabularnewline
ctrl-shift-; & uncomment out line or selected lines (removes ";" at
beginning of every line if found)\tabularnewline
\bottomrule
\end{longtable}

\begin{longtable}[]{@{}ll@{}}
\caption{Editing GenericScore, PythonObject, RhinoObject}\tabularnewline
\toprule
Shortcuts & Description\tabularnewline
\midrule
\endfirsthead
\toprule
Shortcuts & Description\tabularnewline
\midrule
\endhead
ctrl-T & when the text area is focused, ctrl-T runs test on the
soundObject (same as pressing the test button)\tabularnewline
\bottomrule
\end{longtable}
