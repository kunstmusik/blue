\section{Installation}\label{installation}

\subsection{Requirements}\label{requirements}

Blue requires a Java 7 (also known as 1.7) or greater JVM (Java Virtual
Machine). To test to see if you have a JVM installed and what version,
at a command prompt type "java -version". If you see something along the
lines of "command not found" then you need to install a Java Virtual
Machine.

A Java JVM for Linux, Windows, Mac OSX, and Solaris can be found at:

\url{http://www.java.com}

From this page, the correct download for the JRE (Java Runtime
Environment) for your platform should be chosen automatically for you.
(Note: There are two main releases of Java, the JRE, and the JDK (Java
Development Kit). You only need the JRE to run Blue, though if you are
interested in developing in Java you need the JDK.)

\subsection{Installing Blue}\label{installingBlue}

To install Blue, you should download the latest ZIP file or DMG from the
Blue Sourceforge page
\href{http://www.sourceforge.net/projects/Bluemusic}{here}. For OSX
users, download the DMG file which contains a Blue.app. You can
double-click the Application to run, as well as copy it to your
Applications folder to install it.

For Linux and Windows users, download the ZIP file and unzip it. Inside
of the bin folder you will see a "Blue" script for Linux or a "Blue.exe"
file for use on Windows.

Note: After starting Blue, you may want to explore the example projects
and pieces found in the Blue/example folder (or right-click and explore
contents of Blue.app/example if on OSX).

\subsubsection{Platform Specific Notes}\label{platformNotes}

The section below has notes for individual platforms.

Blue uses the right mouse click often to show popup menus. If you do not
have a right mouse button, you can use ctrl-click for all "rt-clicks"
that are mentioned in this documentation.

To make use of the function key shortcuts (F1-F12), you will need to go
into System Preferences, choose Keyboard, then enable "Use all F1, F2,
etc. keys as standard function keys".

For 64-bit systems, you may run into issues when running Blue with the
API enabled where modifying widget values is not reflected in
performance. This is likely due to the Csound Java API being compiled
with SWIG \textless{} 2.0. As of the time of this writing, the version
of Csound in Debian Stable for amd64 is compiled with SWIG 1.3.0, and
does not work with Blue. To work around this, you can install a newer
version of Csound if available (i.e. from a testing repo), or compile
Csound yourself and ensure you are using SWIG version 2.0 or greater.

\subsection{Installing Csound}\label{installingCsound}

Blue is able to interact with Csound either by calling Csound like a
command-line program (classic Blue), or by directly interacting with
Csound via the Csound API. Instructions on setting up Blue for each
method is described below as well as discussion on benefits and
limitations.

\subsubsection{Using Blue with command-line
Csound}\label{commandLineCsound}

This may be considered "classical" Blue usage and interaction with
Csound as this was the method by which Blue ran with Csound for the
first eight years in existance. The way Blue operates in this mode is by
rendering the .Blue project into a temporary CSD file on disk, then
calling Csound with that temporary CSD file in the same way as if you
were on a command-line shell and executing Csound yourself with that
file.

The benefit to this mode of Csound usage is that it is easier to switch
out your version of Csound or use multiple versions of Csound on the
same computer. It is also a little more stable than using the API in
that if Csound crashes for some reason, it won't take down Blue with it.
Also, it may be more performant to use the command-line mode. These
benefits however need to be weighed against the benefits of using the
API, which is described below.

To use the command-line version, one needs to set up the Csound
executable option for Realtime and Disk Render settings in
\protect\hyperlink{programOptions}{Program Options}.

\subsubsection{Using Blue with the Csound API}\label{csoundAPI}

Enabling Blue to use the Csound API when rendering with Csound opens up
the ability to manipulate and edit widget values and automations in
realtime during a render, as well as the ability to use BlueLive on
Windows. Because of its enhancement to the user-experience while
rendering and composing, it is now the preferred method of using Blue
with Csound. Blue should work out-of-the-box with the API if Csound is
installed using the installers provided on SourceForge, or installed via
a package manager if on Linux.

\begin{quote}
\textbf{Note}

Blue currently only works with the API if the version of Csound used is
compiled using 64-bit doubles. (The float version is not currently
supported when using the API.) There are technical difficulties in
supporting two different versions of Csound API in the same build and it
is not known if or when the float build will be supported. For users
interested in using the float build of Csound with Blue, you will need
to run Blue using the command-line Csound mode.

Additionally, the architecture that Csound is compiled for must match
the architecture of the Java runtime you are using. For example, on
Windows, Csound is currently only built for i386 CPU (i.e. 32-bit
Windows) and not x86\_64/amd64 (i.e. 64-bit Windows). In this case, you
will need to run Blue using a 32-bit Java Runtime. For OSX, this is not
an issue as Csound is compiled as a universal binary for both i386 and
x86\_64. On Linux, it is likely that the version of Csound you
install/compile and the Java Runtime that you install will likely be the
same, but if the API does not show as available it may be something to
check.
\end{quote}

The Java API for Csound is split into two parts: the csnd6.jar file as
well as the lib\_jcsound6.so native library (this file is called
\_jcsound6.dll on Windows, and lib\_jcsound6.jnilib on Mac OSX). For
Csound 5, the names use "csnd" and "\_jcsound" instead. Blue comes with
it's own copy of csnd.jar and csnd6.jar; to use the API from Blue it
will need to have access to the native library to work. If the API is
not enabled for use out-of-the-box, the following explains how to setup
the API on different platforms.

Users using the Windows Installer for Csound should use the doubles
version from SourceForge (for Csound 5, the installer has -d in the
name, for Csound 6, the default is the doubls version). After
installing, the installer should setup everything such that Blue should
work with the API. If for some reason it is unable to do so, or you have
compiled Csound yourself and the location of jcsound.dll is different
from where it is installed with the installer, you can modify the
Blue/etc/Blue.conf file to tell Blue where to find the \_jcsound6.dll.
For example, if the directory where jcsound6.dll is located is in
c:\textbackslash{}myCsound, open up Blue/etc/Blue.conf and modify the
default so that it contains:

\begin{verbatim}
default_options="--branding Blue -J-Xms256m -J-Xmx768m -J-Djava.library.path=c:/myCsound"
        
\end{verbatim}

Linux users should install a doubles version of Csound. The version of
Csound found in package repositories should be one compiled for doubles.
After installing Csound and the Java interface for Csound, locate where
lib\_jcsound6.so is and modify the Blue/bin/Blue file. Search for the
lines that contain "-J-Djava.library.path=/usr/lib/jni" and modify
/usr/lib/jni (the default for Debian/Ubuntu-based systems) to the
directory where lib\_jcsound.so is located.

Mac OSX users should use the installer for Csound from Sourceforge. The
installer should install both the float and doubles version of Csound.
The lib\_jcsound.jnilib will be installed into /Library/Java/Extensions.
If you are compiling your own version of Csound, you can remove the
symlink in /Library/Java/Extensions and either symlink your your version
there or copy it into that folder.

To check if the API is enabled, open Blue and open up the Program
Options. This is available from the Blue-\textgreater{}Preferences menu
option on OSX, and from the Tools-\textgreater{}Options menu option
Windows and Linux. In the Blue tab, under both Disk and Realtime Render
settings, there is an option called "Render Method". If Csound 6 is
availble on your system and Blue was able to find it, it will show a
"Csound 6 API" option. If it was not found but Csound 5 was found, then
a "Csound 5 API" option will show. If neither could be found, you will
only have the "Commandline Runner" option which is always available.
