\section{Introduction}

One of the more curious soundObjects in blue is the \emph{Sound
SoundObject}, which allows for the direct writing of instruments in the
score. It's origin came a long time ago now when I was first beginning
to create more soundObjects for blue. I was thinking of what exactly
should be possible for a soundObject, thinking that if someone were to
make a soundObject, they should be able to also embed instruments and
f-tables that they'll know will always be generated and available. The
case which I was thinking about at the time were non-generic
soundObjects like a drum machine, where the soundObject would not only
have a GUI to develop pattern tracks for different drum sounds, but also
would be able to furnish the instruments and f-tables needed to generate
those sounds, the idea being that one could release a soundObject that a
user could use "straight out of the box", no instrument writing or
tables required. That design decision eventually lead me to the
conception of the \emph{Sound SoundObject}, which uses those mechanisms
that were put in place for all soundObjects. Theoretically I was very
fascinated with the possibility of writing instruments within the main
scoring area, mixing note blocks with pure sound blocks. It became
possible to really compose with the sound in a very direct manner;
saying things like "I want a sine wave here with this envelope, then a
triangle wave here, and on top of that i want a processed sample sound
to come in here" now had a direct translation. You didn't need to write
the instrument, then write a note for it, and then have to work with
both of them as separate entities. You could just add a \emph{Sound
SoundObject}block on the timeline, write your sound using standard orc
code, and move it around in time on the timeline. After having thought
through that possibility to express my musical goals by using the
\emph{Sound SoundObject}, I found that it opened up a lot of how I saw
blue's timeline as well as how I thought about ways to work with blue.

As a Csound user or general electronic musician, it might seem strange
to think of directly writing instruments in a score, especially in
context to the existing music tools and musical models that you've
probably worked with. However, by having the \emph{Sound SoundObject},
I've found that interesting possibilities have opened up, and since it's
creation and inclusion into blue it has been used and has played a part
of just about every piece that I've worked on.

Note: It's not that it's impossible to implement this any other way.
Really, it's just a single instrument with a single note, but, that's a
technical issue, an implementation issue. The interesting thing is that
when working with it, it's really a different thing altogether
conceptually. It's the \emph{sound} \emph{soundObject} in the context of
the timeline that makes its usage interesting and useful(well, for me at
least!).

The following sections will go over how to use \emph{Sound SoundObject,}
what happens when it gets processed by blue, as well as some usage
scenarios and patterns that have arose while using it in my own work.

\section{How to Use the Sound SoundObject}

First, insert the \emph{Sound SoundObject} as you would any soundObject
by rt-clicking (for Mac users, hold down the apple key and click) on the
main timeline of the score area and chose "Add New Sound" from the popup
menu.

A \emph{Sound SoundObject} will be inserted wherever you clicked on the
timeline. You'll notice that this soundObject acts and behaves like any
other soundObject: you can move it around in time, drag to change the
duration, change it's name in the soundObject property dialog, and click
on it to edit it. You won't be able to add any noteProcessors to it,
however, but that will be explained more in detail in the "What happens
when it gets processed" section.

If you click on the soundObject to pull up its editor, you'll see a text
box with a default message "insert instrument text here". In the editor
is where you write your instrument definition, but without writing any
"instr 11" or "endin": the number of the instrument and the correct
formatting of the instrument is handled by the soundObject. For a simple
example, you might try:

\begin{verbatim}
 
aout    oscili 30000, 440, 1
        outs aout, aout
\end{verbatim}

where the 1 at the end of the oscili line is an ftable numbered 1,
defined in the tables editor under the tables tab. (For our example,
let's go ahead and definte the ftable as "f1 0 65536 10 1", which is a
sin wave). At this point you might want to try listening your work file
with the play button (assuming you've set up the command line under
project properties to a command line that will output to speaker; for
more information on this, please consult the blue user's manual). What
you should hear is that where you've put your \emph{Sound SoundObject}
on the timeline (i.e. starts at .5 seconds) should play the sound you've
written, in this example case, a sine wave at 440hz at 30000 amp,
playing at equal volume out both left and right channels.

And that's pretty much it! From here you might want to try copying the
block and pasting it a few times, changing some parameters, embellishing
your instrument definitions, etc. Also, you might want to try mixing
this with other soundObject, i.e. write some instruments in the
orchestra manager, then write some genericScore soundObjects to play
those instruments, as well as use some \emph{Sound SoundObjects}
directly on the timeline.

\section{What Happens When It Gets Processed}

If your confused as to what's going on, it'll probably help to know
exactly what happens when blue processes \emph{Sound SoundObjects.}
blue, whenever it goes to create a .CSD file to use with a commandline
or to generatoe out to a file, has different stages of its compilation.
The relevant parts to know here are that all instruments from the
orchestra manager are first generated, but not yet put into the .CSD.
Next, all soundObjects are called to generate any instruments they might
have. This is where the \emph{Sound SoundObject} would generate an
instrument from your text input. The generated instruments from
\emph{Sound SoundObjects}at this point are assigned an instrument
number. After all instruments are generated from soundObjects, all score
text is then generated. At this point, the instrument number assigned in
the earlier pass is now used by the \emph{Sound SoundObject}to generate
a note for your instrument. The note generated by the \emph{Sound
SoundObject}consists of only three p-fields: the instrument number, the
start of the soundObject, and the duration. No other p-fields are
generated (so your instrument should not use any other p-fields).

For example, let's say you have a \emph{Sound SoundObject}with a start
time at 0.5 seconds and a duration of 2 seconds\emph{.}When blue goes to
get it's instrument, let's say it is assigned instrument number 2. The
generated note will be:

\begin{verbatim}
i2 0.5 2
\end{verbatim}

Perhaps the best way to see it is to do the simple example from the "How
to use the Sound SoundObject" section, generate a CSD file (from the
Project menu, select "Generate CSD to file"), and inspect what got
generated. Comparing the soundObject's representation on the timeline as
a sound and seeing how it got generated out might explain things better.

(NOTE: because the \emph{Sound SoundObject} only write out the three
p-fields, using a noteProcessor really doesn't have any purpose, which
is why the \emph{sound soundObject} does not support noteProcessors)

Ultimately in the lowest level implementation, there is a separation of
a note as well as an instrument, but within blue, that separation is
hidden from the user. From the user's point of view, all they have to do
is write their sounds on the timeline, and they don't have to worry
about numbering the instruments or creating notes for that instrument.

\section{Usage Scenarios and Patterns}

Most of the time I've found myself using the \emph{Sound SoundObject} at
the start of a project when I'm creating new sounds for a piece. I find
it's good a prototyping tool, initially working with sounds on the
scoreTimeCanvas (this is the name of java object that is the main score
timeline area). Usually, when I write instruments in the \emph{Sound
SoundObject}, I define all of the i-time variables in a way that will
facilitate easy conversion to full-fledged instruments should I later
want to do so. This is a general instrument writing pattern of mine that
I would do anyways even before I had blue to use. For example, I might
be designing an sound on the timeline with the something like the
following at the top of the text:

\begin{verbatim}
ipch   = cpspch(8.02)
iamp  = ampdb(80)
ispace  = .2
\end{verbatim}

Later, when I get to a point after sketching out some sounds and finding
I like how things are beginning to flow in the piece, I find that i
usually want to start working with the sounds then as instruments. At
this point, I normally convert the \emph{Sound SoundObject} to a
genericScore object (done by rt-clicking on the soundObject and picking
"Convert to Generic Score" from the popup menu), which automatically
takes the the instrument from the soundObject and adds it to the
orchestra under the orchestra manager, and also leaves me with a single
three p-field note, which also shows me what instrument number the
instrument was assigned. After that I'll go to the orchestra manager and
edit the instrument to now take in more p-fields, changing the top text
to something like:

\begin{verbatim}
ipch   = cpspch(p4)
iamp  = ampdb(p5)
ispace  = p6
\end{verbatim}

I usually find myself making two or three different sounds, then copying
a bunch of them and changing a few parameters to try out things in time,
then converting them into instruments. It's a nice separation to have
the ability to work just with sounds at the start of a piece for me, as
really, that's what I'm concerned with at the beginning of a piece,
finding the sounds and initially sculpting the sound space that the
piece will take on. After I find what I'm looking for, it's easy for me
to convert all the sounds into instruments and then proceed from there.

Sometimes when you're wanting to just to build a single sound or
texture, maybe to use as a sound effect in project, you might not really
be thinking in terms of scores, notes, and instruments but rather in
terms of sounds in time. In situations like this, the \emph{Sound
SoundObject} would be the first thing I would use, and might really be
the only soundObject I would use. Notes, as a concept, somtimes really
don't play a part of the musical model for a piece. It's not that you
have all these instruments being played everywhere, but rather you have
sounds going on here and there. It might seem like I'm being a little to
theoretical here, but I really think it does play a part in the work
process.

In the situations I've been in when I've been asked to make a sound for
a friend's website or game, I've found it nice to fire up blue, set the
timeline to a really close-up zoom on time, and just work from there to
craft a sound. I would add a \emph{Sound SoundObject} here and there,
maybe use some global variables so I can make \emph{Sound SoundObjects}
that might just function as an lfo or other control instrument, moving
things around just slightly around in time to sculpt the sound. An
oscillator here, maybe blending it into an fm sound, throw in a noise
generator with some formants and a notch filter sweep...

Sometimes I find myself just making sounds with blue and Csound. It
might be because I'm just curious to try something out, I might be
working on really getting to know a synthesis technique, trying to learn
how to express a sound in my mind or maybe to better train my
imagination to know what a sound will really sound like when I write it
down, etc. Sometimes its just that I want to try out some new opcodes I
haven't really ever used.

It's times like this when I find myself just using the \emph{Sound
SoundObject}, as I'm not interested in the note-instrument paradigm,
it's the furthest thing from my mind. I'm focused on achieving a sound,
or on experimenting to see what is the sound of instrument code I've
just written. And I want the flexibility to add more sounds on the
timeline: I don't want to break my concentration to go and think about
numbering instruments, writing notes, moving the note around in time. I
want to see and work with it all of it in one place., as that's what's
going on in my mind.

It's also great practice too, just writing alot of instruments. I'd
imagine that the \emph{Sound SoundObject} would be a useful tool for a
person new Csound, as it allows just working with instrument code.
(Note: you would still need to know the basics of how Csound works, what
is a CSD, and understand how things in blue map the different parts if
itself to the different parts of a CSD file).

\section{Final Thoughts}

Thanks for reading the tutorial! I hope this tutorial has helped to show
how to use \emph{Sound SoundObject}in blue, as well as helped show some
ways in which you might want to use it.

If you have any comments, suggestions for improving this tutorial, or
questions, please feel free to email me at
\href{mailto:stevenyi@gmail.com.}{\nolinkurl{stevenyi@gmail.com.}}

Thanks and good luck!

Steven
